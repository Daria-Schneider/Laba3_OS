\section{Результаты}

В результате работы была разработана кроссплатформенная программа для межпроцессного взаимодействия через memory-mapped files, успешно функционирующая как в Windows, так и в Linux.

\subsection{Ключевые особенности реализации}

\begin{itemize}
    \item \textbf{Эффективное взаимодействие процессов:} Использование разделяемой памяти позволяет достичь высокой скорости обмена данными между процессами
    \item \textbf{Кроссплатформенность:} Программа использует единый код для различных ОС благодаря системе абстракций в модуле \texttt{crossplatform}
    \item \textbf{Надежная синхронизация:} Реализована система флагов для координации работы процессов без конфликтов доступа к данным
    \item \textbf{Корректное управление ресурсами:} Обеспечено правильное закрытие разделяемой памяти и процессов при завершении работы
\end{itemize}

\subsection{Пример работы программы}

\begin{verbatim}
Lab 3 - Parent process started
Creating child process...
Child process started
File 'output.txt' opened successfully
Enter file name: output.txt
Ready. Enter strings (empty to exit):
> Hello
Result: OK
> world
Result: ERROR: must start with capital letter
> Test
Result: OK
> 
Parent process finished
\end{verbatim}

\noindent\textbf{Содержимое файла output.txt:}
\begin{verbatim}
Hello
Test
\end{verbatim}

\subsection{Эффективность решения}

Программа демонстрирует высокую производительность при обработке строк благодаря использованию технологии memory-mapped files. Прямой доступ к общей области памяти исключает накладные расходы на копирование данных, что делает решение оптимальным для задач интенсивного обмена информацией между процессами.

Все компоненты программы работают стабильно, обеспечивая корректное взаимодействие процессов и надежное выполнение поставленной задачи.