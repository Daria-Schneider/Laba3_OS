\section{Условие}
Родительский процесс создает дочерний процесс. Первой строкой пользователь в консоль
родительского процесса вводит имя файла, которое будет использовано для открытия File с таким
именем на запись. Родительский и дочерний процесс должны быть представлены разными программами.
Родительский процесс принимает от пользователя строки произвольной длины и пересылает их через
memory-mapped files. Процесс child проверяет строки на валидность правилу. Если строка соответствует правилу,
то она выводится в файл, иначе через memory-mapped files выводится
информация об ошибке. Родительский процесс полученные от child ошибки выводит в
стандартный поток вывода.

{\bfseries Цель работы:} Приобретение практических навыков управления процессами в операционных системах семейства Windows и Linux/Unix, а также организация межпроцессного взаимодействия с использованием memory-mapped files. Дополнительной целью являлась разработка кроссплатформенного решения, абстрагирующего особенности системных API.

{\bfseries Задание:} Разработать программу, состоящую из двух процессов — родительского и дочернего, взаимодействующих через memory-mapped files.

Родительский процесс должен:
\begin{itemize}
    \item Создать область разделяемой памяти;
    \item Запрашивать у пользователя имя файла и передавать его дочернему процессу через memory-mapped files;
    \item Принимать от пользователя строки и передавать их дочернему процессу через memory-mapped files;
    \item Получать от дочернего процесса сообщения о результатах обработки строк и выводить их на экран;
    \item Обеспечивать синхронизацию процессов через флаги в разделяемой памяти.
\end{itemize}

Дочерний процесс должен:
\begin{itemize}
    \item Открыть область разделяемой памяти;
    \item Получить от родительского процесса имя файла через memory-mapped files и открыть его для записи;
    \item Принимать строки от родительского процесса через memory-mapped files;
    \item Проверять, начинается ли каждая строка с заглавной буквы;
    \item Если строка начинается с заглавной буквы — записывать её в файл;
    \item Если строка не начинается с заглавной буквы — отправлять сообщение об ошибке родительскому процессу через memory-mapped files;
    \item Завершать работу после получения пустой строки;
    \item Синхронизировать свою работу с родительским процессом через флаги в разделяемой памяти.
\end{itemize}

{\bfseries Технология взаимодействия:} Memory-Mapped Files (Разделяемая память)

{\bfseries Вариант:} 15