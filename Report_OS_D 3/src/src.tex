\section{Метод решения}

Для решения задачи применена архитектура с двумя процессами (родительским и дочерним), взаимодействующими через технологию memory-mapped files (разделяемая память).

\subsection{Основной алгоритм работы}
\begin{enumerate}
    \item \textbf{Инициализация:} Создание области разделяемой памяти для обмена данными между процессами
    \item \textbf{Запуск процесса:} Создание дочернего процесса с установкой каналов стандартного ввода/вывода
    \item \textbf{Передача параметров:} Отправка имени файла через разделяемую память с использованием флагов синхронизации
    \item \textbf{Обработка данных:}
    \begin{itemize}
        \item Родительский процесс читает строки от пользователя и передает через разделяемую память
        \item Дочерний процесс проверяет каждую строку на соответствие критерию (начало с заглавной буквы)
        \item Валидные строки записываются в файл, сообщения об ошибках отправляются через разделяемую память
        \item Синхронизация осуществляется через флаги has\_command, has\_response, has\_filename
    \end{itemize}
    \item \textbf{Завершение работы:} Корректное закрытие разделяемой памяти и процессов при получении пустой строки
\end{enumerate}

\subsection{Особенности реализации}
Для обеспечения кроссплатформенности разработан уровень абстракции, скрывающий различия между API Windows и Unix-систем при работе с memory-mapped files. Реализована поддержка как латинских, так и кириллических символов при проверке заглавных букв.

\section{Описание программы}

Программа реализована в модульном стиле и состоит из четырех основных компонентов.

\subsection{Модуль parent.c}
Отвечает за работу родительского процесса:
\begin{itemize}
    \item Создает общую память для обмена данными
    \item Запускает дочерний процесс
    \item Получает от пользователя имя файла и строки для проверки
    \item Отправляет данные дочернему процессу через общую память
    \item Получает ответы от дочернего процесса
    \item Показывает результаты проверки на экране
    \item Управляет завершением работы программы
\end{itemize}

\subsection{Модуль child.c}
Отвечает за работу дочернего процесса:
\begin{itemize}
    \item Открывает общую память для обмена данными
    \item Получает имя файла от родительского процесса
    \item Открывает файл для записи
    \item Читает строки из общей памяти
    \item Проверяет, начинается ли строка с большой буквы
    \item Записывает подходящие строки в файл
    \item Отправляет сообщения об ошибках, если строка не подходит
    \item Следит за флагами для правильной работы с родительским процессом
\end{itemize}

\subsection{Модуль crossplatform.h/c}
Предоставляет кроссплатформенные абстракции:
\begin{itemize}
    \item \textbf{Структуры данных:} mmap\_file\_t (для memory-mapped files), process\_t (для процессов)
    \item \textbf{Функции работы с разделяемой памятью:} CpMmapCreate, CpMmapOpen, CpMmapClose, CpMmapSync
    \item \textbf{Функции управления процессами:} CpProcessCreate, CpProcessClose
    \item \textbf{Функции межпроцессного взаимодействия:} CpProcessWrite, CpProcessRead
\end{itemize}

\subsection{Модуль stringutils.h/c}
Содержит функции обработки строк:
\begin{itemize}
    \item \texttt{IsCapitalStart()} - проверка начала строки с заглавной буквы
    \item \texttt{TrimNewline()} - удаление символов новой строки
    \item \texttt{CpStringLength()} - определение длины строки
    \item \texttt{CpStringContains()} - проверка наличия подстроки
\end{itemize}

\subsection{Используемые системные вызовы}
\begin{itemize}
    \item \textbf{Windows:} CreateFileMapping, MapViewOfFile, UnmapViewOfFile, CreateProcess, CreatePipe
    \item \textbf{Unix:} shm\_open, mmap, munmap, fork, pipe, dup2
    \item \textbf{Кроссплатформенные:} fopen, fclose, fgets, fprintf, fflush
\end{itemize}

Архитектура программы обеспечивает четкое разделение ответственности между модулями, поддерживает работу в различных операционных средах и использует современные технологии межпроцессного взаимодействия через разделяемую память.