\section{Выводы}

В ходе выполнения лабораторной работы были успешно достигнуты все поставленные цели и решены основные задачи:

\begin{enumerate}
    \item \textbf{Освоена технология memory-mapped files:} На практике применены механизмы работы с разделяемой памятью для организации межпроцессного взаимодействия (\texttt{shm\_open()}, \texttt{mmap()} в Linux и \texttt{CreateFileMapping()}, \texttt{MapViewOfFile()} в Windows)
    
    \item \textbf{Реализовано эффективное межпроцессное взаимодействие:} Организован обмен данными между процессами через общую область памяти, что обеспечило высокую скорость передачи информации по сравнению с традиционными каналами
    
    \item \textbf{Разработана система синхронизации процессов:} Создан механизм координации работы процессов с использованием флагов состояния в разделяемой памяти, предотвращающий конфликты доступа к данным
    
    \item \textbf{Создано кроссплатформенное решение:} Реализована абстракция для работы с memory-mapped files, позволяющая программе функционировать в Windows и Linux без изменения основной логики
    
    \item \textbf{Решены практические задачи:} 
    \begin{itemize}
        \item Организована передача имени файла и строк данных через единую структуру в разделяемой памяти
        \item Реализована проверка строк на соответствие критерию (начало с заглавной буквы)
        \item Обеспечено корректное управление ресурсами разделяемой памяти
        \item Настроена правильная последовательность завершения работы процессов
    \end{itemize}
\end{enumerate}

Работа продемонстрировала преимущества использования memory-mapped files для задач интенсивного обмена данными между процессами. Полученные навыки могут быть применены при разработке высокопроизводительных приложений, требующих эффективной межпроцессной коммуникации и разделения вычислительных задач между независимыми процессами.

\pagebreak